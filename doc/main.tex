\documentclass[a4paper]{article}

\usepackage[utf8]{inputenc}
\usepackage[serbian]{babel}

\usepackage{tikz}
\usepackage{tcolorbox}
\usepackage{amsmath}
\usepackage{amsthm}
\usepackage{amsfonts}	
\usepackage{float}
\usepackage{listings}
\usepackage{fancyhdr}

\begin{document}
    \begin{titlepage}
    \begin{tabular}{ l  c   r  }
        \includegraphics[height=0.1\textwidth, width=0.1\textwidth]{img/uns.png} &
        \begin{tabular}{c}
            \large\textbf{UNIVERZITET U NOVOM SADU} \\
            \large\textbf{FAKULTET TEHNIČKIH NAUKA}  \\
        \end{tabular}
        \includegraphics[height=0.1\textwidth, width=0.1\textwidth]{img/ftn-logo.jpg}
    \end{tabular}
    \vspace*{1cm}
    \begin{flushleft}
        UNIVERZITET U NOVOM SADU \\
        FAKULTET TEHNIČKIH NAUKA \\
        NOVI SAD \\
        Departman za računarstvo i automatiku \\
        Odsek za računarsku tehniku i računarske komunikacije \\
    \end{flushleft}
    \vspace*{2cm}
    \begin{center}
        \LARGE\textbf{ISPITNI RAD}
    \end{center}
    \vspace*{0.25cm}
    \begin{flushleft}
        \begin{tabular}{l l}
            Kandidat:& Lazar Nagulov \\
            Broj indeksa:& SV61/2022 \\
            & \\
            Predmet:&    Objektno orijentisano programiranje 2 \\
            Tema rada:&	Sudoku \\
            & \\
            & \\
            Mentor rada:&    dr Miodrag Đukić
        \end{tabular}
    \end{flushleft}
    \vfill
    \begin{center}
        Novi Sad, januar, 2024.
    \end{center}
\end{titlepage}
    \input{content.tex}
    
    \section{Uvod}
    \pagenumbering{arabic}
    \rhead{1. Uvod}
    \subsection{Sudoku}
    Sudoku je logička zagonetka najčešće u obliku $9 \times 9$ tabele (matrice).
    U prazna polja tabele se upisuju cifre, tako da se svaka broj mora pojaviti
    tačno jednom u svakom redu, svakoj koloni i svakoj $3\times 3$ podmatrici (bloku).
    \begin{figure}[h]
        \centering
        \includegraphics[width=0.5\textwidth, height=0.5\textwidth]{img/sudoku-example.jpg}
        \caption{Primer sudoku zagonetke}
    \end{figure}
    \par Zagonetka ne mora da ima jedno rešenje, ali je standard da ga ima. Primer rešenje zagonetke sa slike 1:
    \begin{figure}[h]
        \centering
        \includegraphics[width=0.5\textwidth, height=0.5\textwidth]{img/sudoku-example-sol.jpg}
        \caption{Primer rešenja sudoku zagonetke}
    \end{figure}
    \subsection{Zadatak}
    Realizovati konzolnu aplikaciju u C++-u (C++17) koja omogućava rešavanje i generisanje sudoku zagonetki. Korisnik unosi dateteke kroz argumente komandne linije. 
    Primer pokretanja programa:
    \par\texttt{./sudoku $input.txt$ $output.txt$}\\
    Argumenti:
    \begin{enumerate}
        \item $input.txt$ - Datoteka iz koje se čita zagonetka.
        \item $output.txt$ - Datoteka u koju se upisuje zagonetka.
    \end{enumerate}
    \par Svaka datoteka sadrži \texttt{jednu} zagonetku, svaki red predstavlja jedan red u tabeli, i svako polje je odvojeno razmakom. Ukoliko datoteke nisu navedene ili ne postoje,
    korisniku se ispisuje način korišćenja programa.
    \par Nakon uspešnog pokretanja programa, prikazuje se početni meni koji nudi opcije:
    \begin{itemize}
        \item \texttt{Generate new Sudoku puzzle} - Generisanja nove zagonetke.
        \item \texttt{Load Sudoku puzzle from file} - Učitavanja zagonetke.
        \item \texttt{Exit} - Izlaska iz igre.
    \end{itemize}
    Generisana zagonetka se upisuje u $output.txt$ datoteku.
    Nakon generisanja ili učitavanja zagonetke, korisnik može da:
    \begin{itemize}
        \item \texttt{Import solution} - Učita rešenje.
        \item \texttt{Solve} - Dopusti programu da reši zagonetku.
        \item \texttt{Exit} - Izlađe iz igre.
    \end{itemize}
    \par Na konzolnoj aplikaciji, posle rešenja koje je generisao program ili učitao iz datoteke, prikazuju se statistički podaci igre, uključujući broj dobro 
    postavljenih polja, broj grešaka i brojač odigranih igara i spisak svih pronađenih grešaka. Nakon završetka igre, korisnik ima opciju da odabere ponovno igranje, 
    što pokreće novu iteraciju igre.
    \newpage

    \rhead{2. Analiza problema}
    \section{Analiza problema}
    \par Glavni problem zadatka je mogućnost rešavanja zagonetke, jer nam je algoritam za rešavanju zagonetke neophodan za generisanje iste. Takođe je potrebna provera
    validnosti unete zagonetke. Prvo ćemo razmotriti probleme rešavanja zagonetke, pa onda problem validacije.
    \par Najčešći način koji se koristi u rešavanju zagonetke je isprobavajući svaka moguća rešenja. Broj mogućnosti da se prazna polja eksponencijalno zavisi od broja praznih polja, pa je
    neophodno da razmotrimo mogućnosti optimizacije algoritma. Sam algoritam je rekurzivan, što može da dovede do 
    prelivanja steka (eng. Stack overflow). Znači da moramo da umanjivo broj mogućih rekurzivnih poziva. Pozitivna strana je sigurno pronalaženje 
    rešenja za svaku zagonetku (ukoliko ono postoji) i što u osnovi ne koristimo dodatne strukture podataka izuzev same tabele. Nažalost, za uvođenje bilo kakvih optimizacija neophodno je
    dodatno zauzeće memorije.
    \par Za potrebe validiranja postavlja se pitane „Šta se ubraja u greške?". Na primer, ako je broj u tabeli duplikat u koloni i u redu, koliko se tačno nalazi grešaka u tabeli?
    U našem slučaju, ukoliko naiđemo na broj koji je duplikat, on se uklanja iz tabele (sem ako nije bio u originalnoj zagonetci, onda se uklanja prethodni broj sa istom vrednošću) i računa kao jedna greška.
    \par Takođe moramo razmotriti mogućnosti generisanje pseudoslučajnih brojeva za generisanje tabele. Primećujemo da korišćenjem obične random funkcije često generiše iste,
    ili veoma slične tabele.
    \newpage

    \rhead{3. Koncept rešenja}
    \section{Koncept rešenja}
    
    \subsection{Algoritmi za rešavanje zagonetke}
    \subsubsection{Obrnuta pretraga}
    Najčešće korišćen algoritam za rešavanje sudoku zagonetke je obrnuta pretraga (eng. Backtracking).
    Ovo je algoritam grube sile (eng. Brute force) koji isprobava sve moguće kombinacije. Dakle, potrebno je da se prođe kroz
    svako polje u tabeli. Ukoliko je polje prazno, upisujemo cifru koja u trenutnoj tabeli ispunjava sva pravila. Nakon upisivanje cifre, rekurzivno pozivamo funkciju 
    - pokušavamo da pronađe rešenje sa novom tabelom. Ukoliko rešenje nije pronađeno, vraćamo se nazad i upisujemo drugu cifru. 
    \par Vremenska složenost ovog algoritma je $\mathcal{O}(n ^ m)$, gde je $n$ dimenzija tabele, a $m$ broj polja koja trebaju da se popune.
    (U našem slučaju je složenost $\mathcal{O}(9^n)$). Minimalan broj polja koja
    moraju biti popunjena je $17$, dakle, u najgorem slučaju se provera $9^{64}$ mogućnosti!
    
    \subsubsection{Optimizacija obrnute pretrage}
    Način na koji možemo optimizovati algoritam obrnute pretrage je da ubrzamo proveru da li se broj može postaviti na zadatoj poziciji. To ćemo postići tako što ćemo pamtiti
    koji broj se našao u redu, koloni i bloku. Za to ćemo koristiti \texttt{std::bitset} iz zaglavlja \texttt{bitset} gde, ako se za $i \in [0,8]$ na $i$-toj poziciji nalazi 1, znači da se cifra $i+1$ nalazi u redu, koloni ili bloku.
    \par Pre samog ulaska u rekurzivnu funkciju obrnute pretrage, moramo proći kroz tabelu i zapisati svaki broj koji se nalazi u tabeli u nizove bitova. Potrebna su 3 niza bitova - 
    \begin{figure}[h]
        \centering
        \includegraphics[width=0.5\textwidth, height=0.5\textwidth]{img/conversion.png}\\
        \texttt{Polje (4,4)}\\
        \begin{tabular}{ l r }
            \texttt{RowSet: } & \texttt{010001101}\\
            \texttt{ColSet: } & \texttt{111100011}\\
            \texttt{BlockSet:}& \texttt{010100110}\\
        \end{tabular}
        \caption{Primer konverzije polja u binaran broj}
    \end{figure}
    za red, kolonu i blok. Za proveru da li je cifru moguće upisati
    koristimo bitnu operaciju ili (eng. bitwise or): 
    \par\texttt{std::bitset<> contain = rows[row] | cols[col] | blocks[block];}\\
    Novi niz bitova \texttt{contain} ima $0$ na $i$-toj poziciji ako je moguće postaviti cifru $i+1$ na poziciju '(row, col)'. Na slici 3 vidimo da se u 
    redu nalaze brojevi 1,3,4,8 (010001101), koloni 7,9,6,2,1,8 (111100011) i u bloku 6,3,2,8 (010100110). Iz ovoga zaključujemo da je u polje (4,4)
    moguće upisati broj 5 (111101111).
    \par Vremenska složenost je i dalje $\mathcal{O}(n^m)$, gde je $n$ dimenzija tabele, a $m$ broj polja koja trebaju da se popune, stim da je su sve provere da li se broj može upisati 
    u polje svedene na $\mathcal{O}(m)$ za razliku od prethodnog algoritma koji ima složenost $\mathcal{O}(nm)$. 
    \par U Tabeli 1. vidimo da se prosečno vreme rešavanja duplo brže. Sam test je sproveden na 1000 tabela koja sadrže 25 popunjenih polja.
    \begin{table}[h]
        \centering
        \begin{tabular}{ |c|c|c|c|}
            \hline
            & Maksimum & Medijana & Srednja vrednost \\
            \hline
            Gruba sila & 6.20147s & 0.001222s & 0.04691s \\
            \hline
            Optimizacija & 2.14567s & 0.00072s & 0.02301s \\
            \hline
        \end{tabular}
        \caption{Optimizacija obrnute pretrage}
    \end{table}

    \subsection{Algoritmi za generisanje zagonetke}
    \begin{figure}[H]
        \centering
        \includegraphics[width=0.25\textwidth, height=0.25\textwidth]{img/diag-fill.jpg}
        $\longrightarrow$
        \includegraphics[width=0.25\textwidth, height=0.25\textwidth]{img/full.jpg}
        $\longrightarrow$
        \includegraphics[width=0.25\textwidth, height=0.25\textwidth]{img/generated.jpg}
        \caption{Primer generisanje zagonetke}
    \end{figure}
    \subsubsection{Koraci u generisanju zagonetke}
    Generisanje zagonetke se može opisati u dva koraka:
    \begin{enumerate}
        \item Popuniti čitavu tabelu slučajnim vrednostima [1, \texttt{BOARD\_SIZE}].
        \item Nasumično obrisati brojeve iz tabele.
    \end{enumerate}
    
    \par Primetimo da se blokovi na glavnoj dijagonali mogu zasebno popuniti jer ne utiču jedan na drugog. Dakle, prvi korak u generisanju se svodi na 
    popunjavanja glavne dijagonale i rešavanja tako generisane zagonetke. Zatim brišemo brojeve iz tabele. Koliko brojava trebamo obrisati zavisi od težine zagonetke
    koju želimo generisati (Tabela 2).
    \begin{table}[H]
        \centering
        \begin{tabular}{ | c | c |}
            \hline
            Težina & Broj popunjenih polja \\
            \hline
            Lako & 32 - 38\\
            \hline
            Srednje & 26 - 31\\
            \hline
            Teško & 22 - 25\\
            \hline
            Veoma teško & 17 - 21\\
            \hline
        \end{tabular}
        \caption{Težina zagonetke u odnosu na broj popunjenih polja}
    \end{table}
    \subsubsection{Generisanje pseudoslučajnih brojeva}
    Za generisanje pseudoslučajnih brojeva koristimo Mersen Tviser algoritam (eng. Mersenne Twister algorithm). 
    Implementacija samog algoritma se nalazi u zaglavlju \texttt{random} (\texttt{std::mt19937}). Algoritam ima ogroman period ($2^{19937}-1$), što znači da se niz
    brojeva neće ponavljati. Seme (eng. seed) se nasumično određujemo koristeći \texttt{std::random\_device}, a za samo generisanje brojeva u određenom intervalu
    koristimo \texttt{std::uniform\_int\_distribution}.
    \par Ovo nam omogućava generisanje nasumičnih brojeva visokog kvaliteta.
    \subsection{Algoritmi za proveru validnosti zagonetke i za brojanje grešaka}
    Provera validnosti unete zagonetke od strane korisnika je jednostavno - proverimo da li u svakom redu, koloni ili bloku postoje duplikati. 
    U slučaju da korisnik unosi rešenje, moramo dodatno da vodimo računa o promeni nepraznih vrednosti iz početne zagonetke.
    \par Svakim pronalaskom pogrešnog unetog broja, izbacujemo isti iz tabele. Tako\-đe vodimo računa da ne izbacimo broj koji je bio prisutan u početnoj tabeli i da isti brojevi nisu izmenjeni u rešenju.
    Pošto se kroz matricu preće sa leva na desno, postoje dve mogućnosti:
    \begin{enumerate}
        \item Naišli smo na broj koji je bio u prethodnoj tabeli i nije validan. Znamo da smo pre njega našli drugi broj koji nije validan zbog njega. Brišemo drugi broj.
        \item Naišli smo na broj koji nije bio u prethodnoj tabeli i nije validan, brišmo ga.
    \end{enumerate}
    Dakle, pored informacije da li se broj pojavio u redu, koloni ili bloku, moramo da znamo i gde se tačno pojavio. Mana ovoga algoritma je favorizovanje brojeva na koje
    se prvo naiđe.
    \newpage
    \rhead{4. Opis rešenja}
    \section{Opis rešenja}
    
    \subsection{Modul glavnog programa (Main)}
    \subsubsection{Funkcija main}
    \texttt{int main(int argc, char** argv);}
    \par Glavna funckija programa. Proverava validnost datoteka prosleđenih preko argumenata komandne linije. 
    Kreira objekat tipa Sudoku i prosleđuje mu imena datoteka.
    \subsubsection{Funkcija za pomoć prilikom pokretanja (Usage)}
    \texttt{void Usage();}
    \par Ispisuje način pokretanja programa u slučaju da potrebni parametri nisu validni.

    \subsection{Modul Sudoku (Sudoku)}
    Glavni modul koji povezuje sve funkcionlanosti iz modula tabele u jednu celinu.
    \subsubsection{Članovi}
    \begin{tabular}{ l l }
        \texttt{const std::string inputFile} & Naziv ulazne datoteke\\
        \texttt{const std::string outputFile} & Naziv izlazne datoteke\\
        \texttt{int currentRound;} & Trenutna runda\\
        \texttt{int correctCount;} & Broj tačnih cifara\\
        \texttt{int wrongCount;} & Broj pogrešnih cifara\\
        \texttt{Board board;} & Sudoku tabela\\
    \end{tabular}
    
    \subsubsection{Enumeracije}
    \texttt{enum Difficulty;}
    \par Određuje težinu Sudoku zagonetke na osnovu broja izbrisanih polja. \\ 
    Vrednost: \texttt{EASY, MEDIUM, HARD, VERY\_HARD}

    \subsubsection{Konstruktor}
    {\parindent0pt
    \texttt{Sudoku(const std::string\& inFile, const std::string\& outFile);}
    }
    \par Glavni parametrizovani konstruktor.\\
    Parametri:
    \begin{itemize}
        \item (\texttt{std::string\&}) inFile - ime datoteke iz koje će se čitati zagonetka ili njeno rešenje.
        \item (\texttt{std::string\&}) outFile - ime datoteke u koju će se upisivati rešenje ili novo generisana zagonetka.
    \end{itemize}

    \subsubsection{Funkcija članica za pokretanje (Run)}
    \texttt{void Run();}
    \par Pokreće aplikaciju i kreira početni meni za korisnika. Nudi korisniku mogu\-ćnost da generiše 
    ili učita zagonetku iz datoteke. Nakon generisanja ili učitavanja zagonetke, korisnik može da izabere način rešavanja.

    \subsubsection{Funcija članica za unos rešenja (SolvingOptions)}
    \texttt{void SolvingOptions();}
    \par Nudi korisniku mogućnost da učita rešenje ili da dopusti programu da sam reši zagonetku. Poziva se  iz \texttt{Run()} funkcije nakon generisanja 
    ili učitavanja zagonetke.

    \subsubsection{Funkcija članica za rešavanje zagonetke (Solve)}
    \texttt{bool Solve();}
    \par Rešava zagonetku koristeći algoritam obrnute pretrage. Rešenje upisuje u $output.txt$ datoteku prosleđenu preko argumentana komandne linije.\\
    Povratna vrednost:
    \begin{itemize}
        \item (\texttt{bool}) True ako je zagonetka uspešno rešena.
    \end{itemize}

    \subsubsection{Funckija članica za proveru rešenja (CheckSolution)}
    \texttt{void CheckSolution();}
    \par Učitava zagonetku iz $input.txt$ datoteke prosleđene preko argumentana komandne linije i proverava koliko ima grešaka. Ispisuje sve greške koje pronađe, njihov broj i broj tačno upisanih brojeva.

    \subsubsection{Funkcija članica za gerenisanje zagonetke (Generate)}
    \text{void Generate(Difficulty difficulty);}
    \par Generiše Sudoku zagonetku sa zadatom težinom. Upisuje generisanu tabelu u $output.txt$ datoteku prosleđenu preko argumentana komandne linije.\\
    Parametri:
    \begin{itemize}
        \item (\texttt{Sudoku::Difficulty}) difficulty - enumeracija koja označava težinu zagonetke.
    \end{itemize}
   
    \subsection{Modul Tabela (Board)}
    Modul koji sadrži pomoćne funkcije potrebne za menjanje trenutne tabele ili provere njene validnosti.
    \subsubsection{Konstante}
    \begin{tabular}{ l l }
        \par\texttt{int BOARD\_SIZE = 9;} & Veličina tabele. \\
        \par\texttt{int BLOCK\_SIZE = 3;} & Veličina bloka. \\
        \par\texttt{int EMPTY = 0;}  & Oznaka za prazno polje. \\
        \par\texttt{char EMPTY\_CHAR = '\_';}  & Oznaka za prazno polje prilikom ispisa.
    \end{tabular}
    
    \subsubsection{Članovi}
    \begin{tabular}{ l l }
        \par\texttt{int board[BOARD\_SIZE * BOARD\_SIZE];} & Niz koji predstavlja tabelu.\\
    \end{tabular}

    \subsubsection{Alijas BitArray}
    \texttt{using BitArray = }\\
    \texttt{std::array<std::bitset<Board::BOARD\_SIZE>, Board::BOARD\_SIZE>;}
    \par Niz od \texttt{Board::BOARD\_SIZE} setova bitova dužine \texttt{Board::BOARD\_SIZE}.

    \subsubsection{Alijas PairArray}
    \texttt{using PairArray = }\\ 
    \texttt{std::array<std::pair<int, int>, BOARD\_SIZE>}
    \par Niz od \texttt{Board::BOARD\_SIZE} pozicija na tabeli.

    \subsubsection{Funckija članica za proveru validnosti tabele (IsValid)}
    \texttt{bool IsValid() const;}
    \par Proverava da li trenutna tabela ispunjava sva pravila sudoka.\\
    Povratna vrednost:
    \begin{itemize}
        \item (\texttt{bool}) True ako je trenutna tabela validna.
    \end{itemize}



    \subsubsection{Funkcija članica za pronalaženje grešaka (CountErrors)}
    \texttt{int CountErrors(const Board\& original) const;}
    \par Prebrojava i ispisuje sve greške u tabeli.\\
    Parametri:
    \begin{itemize}
        \item (\texttt{const Board\&}) original - Originalna tabela pre rešavanja.
    \end{itemize}
    Povratna vrednost:
    \begin{itemize}
        \item (\texttt{int}) Broj grešaka u tabeli.
    \end{itemize}

    \subsubsection{Funkcija članica za dobavljanje broja u tabeli (At)}
    \texttt{int\& At(int row, int col);}
    \par\texttt{const int\& At(int row, int col) const;}
    \par Dobavlja broj na polju '(row, col)'. Proverava granice. Baca \\ \texttt{std::out\_of\_range} u slučaju da
    su uneti brojevi nevalidni.\\
    Parametri:
    \begin{itemize}
        \item (\texttt{int}) row - Red u kome se polje nalazi.
        \item (\texttt{int}) col - Kolona u kome se polje nalazi.
    \end{itemize}
    Povratna vrednost:
    \begin{itemize}
        \item (\texttt{int\&}) Referenca na broj u tabeli.
    \end{itemize}
    
    \subsubsection{Statična funkcija članica za dobavljanja blocka (GetBlock)}
    \texttt{static inline int GetBlock(int row, int col);}
    \par Vraća blok u kome se nalazi polje '(row, col)'.\\
    Parametri:
    \begin{itemize}
        \item (\texttt{int}) row - Red u tabeli.
        \item (\texttt{int}) col - Kolona u tabeli.
    \end{itemize}
    Povratna vrednost:
    \begin{itemize}
        \item (\texttt{int}) Broj bloka u kome se nalazi polje (row, col)
    \end{itemize}

    \subsubsection{Funkcija članica za generisanje elemenata na glavnoj dijagonali (GenerateDiagonal)}
	\texttt{void GenerateDiagonal();}
	\par Generiše nasumično brojeve na glavnoj dijagonali.

    \subsubsection{Funcija članica za generisanje ostalih elemenata (GenerateOther)}
    \texttt{bool GenerateOther(int row, int col);}
    \par Rekurzivno generiše nasumično brojeve koji se ne nalaze na glavnoj dijagonali.\\
	Parametri:
    \begin{itemize}
        \item (\texttt{int}) row - Početan red (uglavnom 0).
        \item (\texttt{int}) col - Početna kolona (uglavnom 0).
    \end{itemize}
    Povratna vrednost:
    \begin{itemize}
        \item (\texttt{bool}) Zaustavlja rekurzivno generisanje tabele.
    \end{itemize}

    \subsubsection{Funkcija članica za brisanje elemenata iz tabele (RemoveNumber)}
	\texttt{void RemoveNumber(int count);}
    \par Nasumično briše $count$ brojeve iz tabele.\\
    Parametri:
    \begin{itemize}
        \item (\texttt{int}) count - broj elemenate koliko se briše iz tabele.
    \end{itemize}

    \subsubsection{Funkcija članica za brojanje praznih polja (CountEmpty)}
    \texttt{int CountEmpty() const;}
    \par Broji prazna polja u tabeli.\\
    Povratna vrednost:
    \begin{itemize}
        \item (\texttt{int}) Broj praznih polja u tabeli.
    \end{itemize}

    \subsubsection{Funkcija članica koja implementira algoritam obrnute pretrage (Backtrack)}
    {\parindent0pt
    \texttt{bool Backtrack(BitArray\& rSet, BitArray\& cSet, BitArray\& bSet);}
    }
    \par Funkcija implementira algoritam obrnute pretrage. Rekurzivno popunjava tabelu. Poziva se iz funkcije \texttt{Solve()} nakon generisanja pomoćnih nizova bitova.
    Parametri:
    \begin{itemize}
        \item (\texttt{BitArray}) rSet - Pomoćni niz koji prati pojavljivanje brojeva u redovima.
        \item (\texttt{BitArray}) cSet - Pomoćni niz koji prati pojavljivanje brojeva u kolonama.
        \item (\texttt{BitArray}) bSet - Pomoćni niz koji prati pojavljivanje brojeva u blokovima.
    \end{itemize}
    Povratna vrednost:
    \begin{itemize}
        \item (\texttt{bool}) Zaustavlja rekurziju kada pronađe rešenje.
    \end{itemize}

    \subsubsection{Funkcija članica za pronalaženje prvog praznog polja (FindEmpty)}
	\texttt{bool FindEmpty(int\& row, int\& col) const;}
    \par Pronalazi prvo prazno polje od pozicije '(row, col)'. Prazno polje se nalazi u $row$ i $col$ promenljivi nakon završetka funkcije.\\
    Parametri:
    \begin{itemize}
        \item (\texttt{int\&}) row - referenca na početan red koji se pretražuje.
        \item (\texttt{int\&}) col - referenca na početnu kolonu koja se pretražuje.
    \end{itemize}
    Povratna vrednost:
    \begin{itemize}
        \item True ako postoji prazno polje.
    \end{itemize}

    \subsubsection{Funckija članica za brisanje tabele (Clear)}
    \texttt{void Clear();}
    \par Postavlja sva polja u tabeli na 0.

    \subsubsection{Funkcija za proveru da li je broj duplikat (IsDuplicate)}
    \texttt{bool Board::IsDuplicate(int row, int col, PairArray\& buff);}\\
    Parametri:
    \begin{itemize}
        \item (\texttt{int}) row - Red u kome se polje nalazi.
        \item (\texttt{int}) col - Kolona u kome se polje nalazi.
        \item (\texttt{PairArray}) buff - Niz koji čuva pozicije prethodno pronađenih brojeva.
    \end{itemize}
    Povratna vrednost:
    \begin{itemize}
        \item (\texttt{bool}) True ako je broj duplikat.
    \end{itemize}

    \subsubsection{Funkcija članica za proveru poteza (IsPossibleMove)}
    \texttt{bool IsPossibleMove(int row, int col, int number) const;}
    \par Proverava da li postavka datog broja na datu poziciju je validan potez.
    Parametri:
    \begin{itemize}
        \item (\texttt{int}) row - red u tabeli
        \item (\texttt{int}) col - kolona u tabeli
        \item (\texttt{int}) number - broj koji se pokušava staviti
    \end{itemize}
    Povratna vrednost:
    \begin{itemize}
        \item (\texttt{bool}) - true ako je moguće postaviti broj, false ako nije.
    \end{itemize}

    \subsubsection{Funkcija članica za popunjavanje blokova (FillBlock)}
    \texttt{void FillBlock(int row, int col);}
    \par Rekurzivno generiše nasumično brojeve u bloku.\\
    Parametri:
    \begin{itemize}
        \item (\texttt{int}) row - početni red (gornje levo polje u bloku).
        \item (\texttt{int}) col - početna kolona (gornje levo polje u bloku).
    \end{itemize}

    \subsubsection{Operatori upisa i ispisa}
    \par\texttt{std::istream\& operator>>(std::istream\& in, Board\& b);}
    \par Operator za čitanje tabele iz datoteke.
    \par\texttt{std::ostream\& operator<<(std::ostream\& out, const Board\& b);}
    \par Operator za ispisivanje tabele na standarni izlaz - konzolu.
    \par\texttt{std::ofstream\& operator<<(std::ofstream\& out, const Board\& b);}
    \par Operator za upisivanje tabele u datoteku. 

    \subsubsection{Operator dobavljanja}
    \par\texttt{int\& operator()(int row, int col);}
    \par\texttt{const int\& operator()(int row, int col) const;}
    \par Operator za dobavljanje broj na poziciji '(row, col)'.
    
    \newpage
    \rhead{5. Testiranje}
    \section{Testiranje}
    \subsection{Način testiranja}
    Implementacija se testira pomoću modula \texttt{Test}. Napravljeni su posebni testni skupovi (skupovi testnih slučajeva) za svaki modul, 
    a svaki testni skup sadrži testove (testne slučajeve, test cases) raznih segmenata tog modula - funkcija i struktura, kao i testiranje njihovih međusobnih odnosa.
    \par Testovi se pokreću komandom \texttt{./sudoku -test}.
    \subsection{Funkcija za pokretanje}
    \texttt{void TestRun();}
    \par Pokreće sve testove.
    \subsection{Pomoćna funkcija za poređenje tabela}
    {\parindent0pt
    \texttt{int CheckErrors(const std::string\& org, const std::string\& cmp);}
    }
    \par Poredi originalnu i rešenu zagonetku kako bi prebrojao greške.\\
    Parametri:
    \begin{itemize}
        \item (\texttt{std::string\&}) org - Ime originalne datoteka gde se nalazi zagonetka.
        \item (\texttt{std::string\&}) cmp - Ime datoteke gde se nalazi rešenje prethodne zagonetke.
    \end{itemize}
    Povratna vrednost:
    \begin{itemize}
        \item (\texttt{int}) Broj pronađenih grešaka.
    \end{itemize}
    \subsection{Testni skupovi}
    {\parindent0pt
        Skup testova za verifikaciju tabele.
        \begin{itemize}
            \item $invalid\_row.txt$
            \item $invalid\_column.txt$
            \item $invalid\_block.txt$
            \item $valid.txt$
        \end{itemize}
        \par Skup testova za proveru korisnikovog rešenja:
        \begin{itemize}
            \item $original\_changed.txt$ i $error\_changed.txt$
            \item $original\_column.txt$ i $error\_column.txt$
            \item $original\_row.txt$ i $error\_row.txt$
            \item $original\_block.txt$ i $error\_block.txt$
            \item $original\_all\_[i].txt$ i $error\_all\_[i].txt$, $i\in[1,3]$
        \end{itemize}    
    }
    \subsection{Testiranje validnosti korisnikove zagonetke (IsValid)}
    \subsubsection{Validna zagonetka}
    Test učitava validnu zagonetku.
    \subsubsection{Nevalidna kolona}
    Test učitava zagonetku sa tačno jednim duplikatom u koloni. 
    \subsubsection{Nevalidan red}
    Test učitava zagonetku sa tačno jednim duplikatom u redu.
    \subsubsection{Nevalidan blok}
    Test učitava zagonetku sa tačno jedinim duplikato u bloku.

    \subsection{Testiranje validnosti korisnikovog rešenja (CountErrors)}
    \subsubsection{Izmena nepraznog početnog polja}
    Test učitava početnu tabelu i rešenu tabelu sa izmenjenim nepraznim poljima. 
    \subsubsection{Nevalidne kolone}
    Test učitava početnu tabelu i rešenu tabelu sa nevalidnim kolonama. 
    \subsubsection{Nevalidni redovi}
    Test učitava početnu tabelu i rešenu tabelu sa nevalidnim redovima.
    \subsubsection{Nevalidni blokovi}
    Test učitava početnu tabelu i rešenu tabelu sa nevalidnim blokovima.
    \subsubsection{Sve izmene}
    Test učitava početnu tabelu i rešenu tabelu koja sadrži duplikat u redu, duplikat u koloni, duplikat u bloku i izmenjeno neprazno polje.
    \subsection{Testiranje generisanja i rešavanja tabela (Generate i Solve)}
    Test generiše 100 tabela i pokušava da ih reši.

    \newpage
    \rhead{6. Uočeni problemi i ograničenja}
    \section{Uočeni problemi i ograničenja}
    Prilikom testiranja, uočeni su sledeći problemi i ograničenja. Navedeni su i moguća rešenja.
    \begin{enumerate}
        \item Algoritam za rešavanje može da uzima (ali veoma retko) mnogo vremena za minimalno popunjene tabele (17 popunjenih polja).
        \item Generisanje tabele ne garantuje jedinstvenost rešenja zagonetke.
        \item Brojenje grešaka favorizuje brojeve na koje prvo naiđe (uvek će se brisati duplikat koji ima veće pozicije).
    \end{enumerate}
    \par Rešenje problema jedan je da uvek generišemo tabelu koja ima više od 17 popunjenih polja. Kroz testiranje primećeno je da se tako otklanja ovaj nedostatak, odnosno da se
    mnogo ređe pojavljuje. Takođe je moguće optimizovati brzinu rešavanja koristeći probalističke algoritme poput Knutovog algoritma iks (eng. Knuth's Algorithm X). Primetimo da navedeni algoritam
    takođe koristi algoritam obrnute pretrage i rekurziju.
    \par Rešenje problema dva je provera jedinstvenosti rešenja nakon obrisanog svakoga broja, što dodatno usporava samo generisanje tabele.
    \par Rešenje problema tri je redefinisanje greške ili uvođenje nasumičnosti.
    \newpage
    \rhead{7. Zaključak}
    \section{Zaključak}
    Napravljen je i verifikovan jedan koncept za realizaciju datog problema. Optimizovano je traženje i generisanje rešenje i date su ideje za 
    dodatne optimizacije. Iako je vrednost veličine tabele 9, moguće je generisati i rešavati i veće tabele, izmenom konstanti \texttt{BOARD\_SIZE} i \texttt{BLOCK\_SIZE}, stim
    da važi $\texttt{BLOCK\_SIZE} \in \mathbb{N} \land \texttt{BLOCK\_SIZE} = \sqrt{\texttt{BOARD\_SIZE}}$, odnosno veličina tabele mora biti potpun kvadrat da bi zagonetka imala smisla.
    \par Rezultati merenja iz Tabele 1 pokazuju znatno ubrzanje, naručito za tabele koje imaju više praznih polja. Ovo je manje primetno za tabele koje imaju 
    manji broj praznih polja, jer je veća verovatnoća da uobičajna pretraga brzo pronađe duplikate. 
    \newpage
\end{document}